\documentclass{article}
\usepackage[utf8]{inputenc}
\usepackage[brazilian]{babel}
\usepackage{tikz, tikzpeople}
\usepackage{amsmath, amsthm, amsfonts}
\usepackage{xcolor,colortbl}
\usepackage{tgbonum}
\usetikzlibrary{arrows, arrows.meta, snakes}

\newcommand{\mc}[2]{\multicolumn{#1}{c}{#2}}
\definecolor{Gray}{gray}{0.85}
\definecolor{LightGray}{gray}{0.90}

\newcolumntype{a}{>{\columncolor{Gray}}c}
\newcolumntype{b}{>{\columncolor{white}}c}

\newtheorem*{definition}{Definição}

\title{Protocolo Diffie–Hellman para Troca de Chaves}
\author{Jader Martins\\
        Igor Figueira}
\date{\today}

\begin{document}

\maketitle

\begin{abstract}
    O esquema de troca de chaves proposto por Whitfield Diffie e
    Martin Hellman em 1976 é protocolo para realizar
    o intercambio de chaves criptográficas por meio de um canal público
    de forma segura\cite{Diffie}. Este é um dos primeiros protocolos de
    chave publica, foi originalmente conceitualizado por Ralph Merkle
    e sua descrição formal apresentada posteriormente pelos autores que
    nomeam o protocolo \cite{Merkle}.
\end{abstract}

\section{Introdução}%
\label{sec:introducao}
A comunicação secreta é o estabelecimento de um meio de comunicação entre pares
tal que uma terceira parte, também chamado adversário, tenha acesso mínimo ao
conteúdo que esta sendo intercambiado nesse meio.

Tal atividade e seu aprimoramento datam da grécia
antiga\cite{katz2014introduction}, em grande parte, esses segredos eram
estabelecidos pela substituição trivial de letras, cifração, da mensagem em que
\textit{apriori} eram estabelecidas as regras de substituição e como seria recuperado a
mensagem original. Esses esquemas posteriormente apresentaram diversas
limitações, pois conhecendo o esquema, se tornava trivial a decifração da
mensagem.

Com o surgimento de máquinas automáticas e subsequentemente computadores, a
comunicação segura precisou ser reinventada\cite{singh1999code}, tais mecanismo
permitiam facilmente quebrar esquemas antigos porém propiciaram novas formas de
se codificar mensagens secretas\cite{Diffie}.

Mesmo com o enorme poderio que tais máquinas propiciaram existem limites do ponto
de vista prático, fatores físicos, teóricos e tecnológicos a respeito da computação
impõe limites em um dado momento do que pode ou não ser viável de ser feito por
um computador, isso inclui a criptografia.

Existe um esquemas criptográficos que é seguros mesmo contra um adversário que tenha
um poder computacional infinito, esse é chamado de segredo perfeito
\cite{katz2014introduction}, o qual foi introduzido por Frank Miller em
1882\cite{frankm} e posteriormente provada sua invulnerabilidade teórica
por Claude Shannon\cite{shannon1949communication}.

De forma breve, o protocolo requere gerar um número puramente aleatório do tamanho das
mensagens trocadas, intercambiar chaves sem vazamentos e utiliza-las apenas uma vez\footnote{Uma descrição detalhada do segredo perfeito pode ser encontrado na
seção \ref{sub:segredo_perfeito}}.
Diversas formas para os diferentes critérios foram tentadas, porém usavam de
protocolos caros e complexos\cite{matsumoto1987key,Merkle}.

Tal ritual seria impraticável para grupos que detinham pouco poder econômico,
surgindo então a busca por formas praticáveis de se distribuir chaves para o
estabelecimento de comunicação segura, dessa busca surgiram os trabalhos de
Ralph Merkle, Whitfield Diffie e Martin Hellman.

O protocolo Diffie-Hellman forneceu a primeira solução viável ao problema
de distribuição de chaves, permitindo duas partes, que inicialmente não se
conhecem ou trocaram qualquer material, estabelecer um segredo compartilhado
(uma chave de encriptação) em um canal aberto, essa chave pode então
subsequentemente ser usada para criptografia simétrica. A segurança se
estabelece pela intratabilidade do problema de
Diffie-Hellman\cite{katz1996handbook} e computação de logaritmos
discretos\footnote{A dificuldade de se computar logaritmos discretos é
apresentada na seção \ref{sub:logaritmo_discreto}}.

\section{O Protocolo}%
\label{sec:o_protocolo}

A versão básica do protocolo, provê proteção na forma de gerar uma
chave em comum entre os pares, de tal forma que um adversário passivo não a obtenha,
porém o protocolo não garante proteção contra um adversário ativo, esse capaz
de interceptar, modificar ou injetar mensagens.Nenhuma das partes tem
garantias da identidade da fonte da mensagem recebida ou da parte que saberá
a chave resultante\cite{katz1996handbook}, por isso tal protocolo falha em
resistir a ataques de um adversário ativo como será apresentado na seção
\ref{sec:vulnerabilidade_protocolo}.

Posteriormente o protocolo foi aprimorado para resistir a adversários ativos. Para que isso seja feito, a chave gerada por cada uma das partes é encriptada utilizando a chave privada de cada uma das partes. Dessa maneira, quando o receptor recebe a mensagem, ele utiliza a chave pública do emissor para verificar se quem enviou aquela mensagem realmente é quem diz ser. Dessa maneira é evitado o problema de ataques como \textit{man-in-the-middle}.

\subsection{Exemplo Ilustrativo}
\label{sub:exemplo_ilustrativo}
Para o exemplo ilustrativo, supomos que dois entes, Alice e Bob, querem pintar
seus quartos de uma mesma cor sem que Charlie, que busca impedir essa amizade,
possa descobrir tal cor, figura \ref{fig:abc}. Analogamente, as cores são
chaves, e a separação das cores, tem alto custo, o que em condições ideáis
inviabiliza Charlie de obter as cores secretas e a cor compartilhada.

\begin{figure}[htpb]
    \centering
    \begin{tikzpicture}[every node/.style={minimum width=1.5cm}]
\node[alice] (alice) {Alice};
\node[bob, right= 4cm of alice, mirrored] (bob) {Bob};
\draw[>=stealth',<->,thick] (alice) -- coordinate[near start, label=below:COR] (aux) (bob);
\node[charlie, mirrored, above right=-5mm and 1cm of alice,label=above:Charlie] (charlie) {};
\draw[->, thick] (charlie)--(aux);
\end{tikzpicture}

    \caption{Charlie ouvindo o dialogo de Alice e Bob.}%
    \label{fig:abc}
\end{figure}

O processo é iniciado com Alice e Bob, primeiramente eles definem
uma cor em comum arbitrária, no caso da figura \ref{fig:diagram}, rosa, essa
a qual não é preciso ser mantida em segredo de Charlie. Em seguida cada um deles
escolhe uma cor secreta para si, no exemplo, Alice escolhe amarelo e Bob
escolhe azul, então essas cores são combinadas as cores que ambos compartilham,
rosa. Para Alice é gerada a cor laranja e para Bob é gerada a cor roxa.

\begin{figure}[htpb]
    \centering
    \begin{tikzpicture}[node distance=0.7cm]

\tikzstyle{square}=[draw,minimum size=20pt]
\tikzstyle{wave}=[draw,-latex',snake=snake,segment amplitude=.4mm,segment length=3mm,line after snake=1.5mm]
\tikzstyle{symbol}=[node distance=0.7cm]
\tikzstyle{arrow}=[arrows={{Latex[scale=0.5]}-}, thick]

\node [square, fill=red!25, label=above:Alice]   (b1) at (-2.5,0) {};
\node [square, fill=red!25, label=above:Bob]  (b2) at (2.5,0) {};

\path (b1) --  node [midway]{Cor em comum} (b2);

\node [symbol]  (p1) [below of=b1] {+};
\node [symbol]  (p2) [below of=b2] {+};

\node [square, fill=yellow!25](b3) [below of=p1] {};
\node [square, fill=blue!25](b4) [below of=p2] {};

\path (b3) --  node [midway]{Cores secretas} (b4);

\node [symbol]  (p3) [below of=b3] {=};
\node [symbol]  (p4) [below of=b4] {=};

\node [square, fill=orange!50](b5) [below of=p3] {};
\node [square, fill=purple!50](b6) [below of=p4] {};

\node [symbol]  (p5) [below of=b5] {};
\node [symbol]  (p6) [below of=b6] {};

\node [square, fill=purple!50](b7) [below of=p5] {};
\node [square, fill=orange!50](b8) [below of=p6] {};

\path [wave](b5) --  node [above=1.5em] {Transporte público} (b8);
\path [wave](b6) --  node [text width=3cm,below=1em] {(separação das cores é cara)} (b7);

\node [symbol]  (p7) [below of=b7] {+};
\node [symbol]  (p8) [below of=b8] {+};

\node [square, fill=yellow!25](b9) [below of=p7] {};
\node [square, fill=blue!25](b10) [below of=p8] {};

\path (b9) --  node [midway]{Cores secretas} (b10);

\node [symbol]  (p9) [below of=b9] {=};
\node [symbol]  (p10) [below of=b10] {=};

\node [square, fill=brown!75](b9) [below of=p9] {};
\node [square, fill=brown!75](b10) [below of=p10] {};

\path (b9) --  node [midway]{Segredo compartilhado} (b10);
\end{tikzpicture}

    \caption{Exemplo ilustrativo do protocolo.}%
    \label{fig:diagram}
\end{figure}

Aqui está a parte essencial do processo, essa terceira cor é trocada entre
Alice e Bob em um meio público, por exemplo, no parquinho onde Charlie também
brinca. Mesmo que Charlie saiba a cor inicial, rosa, e a cor que foi trocada no
parquinho, laranja e roxo, Charlie não conseguirá obter as cores secretas de
Alice e Bob. Alice obtendo a cor de Bob e Bob de Alice, esses a adicionam sua
cor secreta inicial, gerando assim a cor marrom identica para ambos.

No cenário ideal, apenas Bob e Alice terão a cor marrom em comum, podendo assim
usa-las exclusivamente, sem que Charlie a descubra. Voltando a analogia inicial,
a cor rosa seria uma chave compartilhada públicamente entre Alice e Bob, as
cores secretas seriam as chaves exclusivas de Alice e Bob, o valor resultante
da operação de ambos poderia ser compartilhado em um canal qualquer que a
construção para a chave final ainda dependeria das chaves exclusivas.

A cor marrom, por fim, seria a chave para encriptar as mensagens trocadas que
apenas Alice e Bob teriam, dado que seria computacionalmente inviável descobrir
as chaves secretas.


\subsection{Exemplo descritivo}%
\label{sub:exemplo_descritivo}
A implementação original do protocolo\cite{Diffie}, usa as propriedades da
aritmética modular, sejam os inteiros coprimos a $n$ do conjunto
${0, 1, ..., n-1}$ de $n$ inteiros positivos formam um grupo sob a
multiplicação modulo $n$, chamado \textit{grupo multiplicativo de inteiros
modulo n}. Se $n$ for primo, $m$ é uma raiz primitiva modulo $n$. Esses dois
valores são escolhidos dessa forma para garantir que o segredo compartilhado
resultante pode assumir qualquer valor entre $1$ e $n-1$. Usaremos letra
latinas para publico e gregas para privado:

\begin{itemize}
    \item Alice e Bob publicamente concordam em usar $\boldsymbol{n = 23}$ e
        $\boldsymbol{m = 5}$ (raiz primitiva 5 modulo 23), como chaves.
    \item Alice escolhe uma chave secreta $\alpha=4$, e então manda para Bob o
        resultado de $A = {m}^{\alpha}\bmod n$.
        \begin{equation}
            A = 5^4\bmod23 = 4
        \end{equation}
    \item Bob escolhe uma chave secreta $\beta=3$, e então manda para Alice o
        resultado de $B = {m}^{\beta}\bmod n$.
        \begin{equation}
            B = 5^3\bmod23 = 10
        \end{equation}
    \item Alice computa o segredo compartilhado ($\omega$), $B^\alpha \bmod n$.
        \begin{equation}
            \omega = 10^4\bmod23 = 18
        \end{equation}
    \item Bob computa o segredo compartilhado ($\omega$), $A^\beta \bmod n$.
        \begin{equation}
            \omega = 4^3\bmod23 = 18
        \end{equation}
    \item Por fim Alice e Bob compartilham a mesma chave secreta $\omega = 18$.
\end{itemize}

Ambos chegam a mesma chave $\omega$, porque, sobre $\bmod n$, segue que,

\begin{equation}
    A^{\beta}\bmod n = m^{\alpha\beta}\bmod n = m^{\beta\alpha}\bmod n = B^{\alpha}\bmod n
\end{equation}

De forma simplificada,

\begin{equation}
    (m ^\alpha \bmod n)^\beta \bmod n = (m^\beta \bmod n)^\alpha \bmod n
\end{equation}

Nesse esquema é visto que a chave $\alpha$ não é conhecido por ninguém alem de Alice,
a chave $\beta$ não é conhecida por ninguém além de Bob e a chave $\omega$ não
é conhecida por ninguém além do par Alice-Bob, na tabela \ref{tab:alicebob} é
apresentada uma descrição detalhada dos níveis de acesso das chaves.

\begin{table}[ht]
    \centering
    \caption{Estagios publicos/secretos das chaves trocadas.}
    \label{tab:alicebob}
    \begin{tabular}{|c|c|c|c|c|c|c|}
    \hline
    \rowcolor{Gray}
    \multicolumn{2}{|c|}{Alice} &  & \multicolumn{2}{c|}{Bob} \\ \hline
       Secreto      &Público   &Envia           &Público   &Secreto   \\ \hline
       \rowcolor{LightGray}
       $\alpha$       &n, m      &n, m $\rightarrow$&          &$\beta$     \\ \hline
       $\alpha$       &n, m, A   &A $\rightarrow$   &n, m      &$\beta$     \\ \hline
       \rowcolor{LightGray}
       $\alpha$       &n, m, A   &B $\leftarrow$    &n, m, A, B&$\beta$     \\ \hline
       $\alpha$,$\omega$&n, m, A, B&                &n, m, A, B&$\beta$,$\omega$\\ \hline
    \end{tabular}
\end{table}

\section{Vulnerabilidade do Protocolo}
\label{sec:vulnerabilidade_protocolo}
Como foi apontado, o Diffie-Hellman em sua forma original, não resiste a ataques de um adversário ativo, \textit{man-in-the-middle}, aqui apresentamos como um adversário que pode interceptar, modificar ou injetar mensagens no meio de comunicação poderia ter acesso as mensagens que se utilizam da versão original desse protocolo.

\section{Aprimoramentos do Protocolo}
\label{sec:aprimoramentos_protocolo}
Até o momento mencionamos apenas chaves simétricas, aquela ao qual ambos, Alice e Bob compartilham da mesma chave, para apresentar uma forma de resistir a ataques ativos iremos introduzir um exemplo de chave assimétrica, também conhecida como chave publico-privada.

\subsection{Chaves Assimétricas}
\label{sub:chaves_assimetricas}

\subsection{Modelo de Correção}
\label{sub:modelo_correcao}
Chaves assimétricas poderiam ser usadas como substituição da Diffie-Hellman, porém estás não seriam chaves por seção, mas sim durante todo o histórico, ou seja, caso um
adversário obtenha uma das chaves privadas, este poderá ter acesso a todo o histórico
de comunicação daquela chave, outra desvantagem de tais chaves é que tem um alto custo computacional e para comunicações instantâneas não oferece uma experiência agradável devido sua latência.

Com os problemas apresentados pelas chaves-assimétricas, é visível ainda a necessidade
de usar chaves-simétricas para comunicação segura e o protocolo Diffie-Hellman ainda
se mostra eficaz.

%% descrever

Embora chaves assimétricas adicionem uma camada a mais de proteção na momento de
se estabelecer a chave, inviabilizando o ataque apresentado pela secção anterior,
ainda existem outras possibilidades de ataque para esse tipo de criptografia
(assimétrica), que vai além do escopo desse trabalho, mas devem ser consideradas
pelo leitor que fará uso de tal solução.

\section{Conclusão}%
\label{sec:conclusao}
% falar que o diffie é muito usado (TLS, HTTPS, etc), ressaltar problemas e vantagens e concluir argumentando o problema importante que ele resolveu;;;; 


\bibliographystyle{unsrt}
\bibliography{sample}


\section{Notas}%
\label{sec:notas}
Nesta seção são apresentados topicos externos que estão relacionados a
viabilidade e implementação do protocolo Diffie-Hellman.

\subsection{Logaritmo Discreto}%
\label{sub:logaritmo_discreto}
Em teoria dos números, o logaritmo discreto de $\log_{b}{a}$ é algum inteiro
$k$ tal que $b^k=a$. Logaritmos discretos são rapidamente computáveis em apenas
alguns casos especiais. Entretanto, nenhum método eficiente é conhecido para
a computação do caso geral. Muitos algoritmos de criptografia residem na
dificuldade intrínseca da reversão em k e são escolhidos de forma que não se
situem nos casos de solução eficiente.

\begin{definition}
    Seja $G$ um grupo que tem por operador a multiplicação e por elemento
    identidade $1$. Sendo $k$ um elemento de G para qualquer
    $k \in \mathbb{N}$ a expressão $b^k$ denota o produto de $b$ por ele mesmo
    $k$-vezes:

    \[
    b^{k}=\underbrace{b\cdot b\cdots b}_{k\text{vezes}}
    \]
    Similarmente, $b^{-k}$ denota o produto de $b^{-1}$ com si mesmo $k$-vezes.
    Para $k=0$, a $k$-esima potencia é a identidade, $b^0=1$. Seja $a$ também
    um elemento de G. Um inteiro k, resolve a equação $b^k=a$ é definido como
    logaritmo discrete, também escrito da forma $k=\log_b{a}$.
\end{definition}

%%% EXPLICAR PORQUE DIFICIL COMPUTAR O K

\subsection{Segredo Perfeito}%
\label{sub:segredo_perfeito}
O esquema de segredo perfeito, também conhecido (em inglês) como, \textit{
one-time pad}, é uma técnica de encriptação que não pode ser quebrada, mas
requere o uso de uma chave \textit{one-time} pre-compartilha, que tenha o mesmo
tamanho ou mais que a mensagem que está sendo compartilhada.

A chave \textit{one-time} é uma chave usada uma única vez e que tem seu
processo de geração puramente aleatório.

%%%% In this technique, a plaintext is paired with a random secret key (also referred to as a one-time pad). Then, each bit or character of the plaintext is encrypted by combining it with the corresponding bit or character from the pad using modular addition. If the key is (1) truly random, (2) at least as long as the plaintext, (3) never reused in whole or in part, and (4) kept completely secret, then the resulting ciphertext will be impossible to decrypt or break.[1][2] It has also been proven that any cipher with the property of perfect secrecy must use keys with effectively the same requirements as OTP keys.[3] Digital versions of one-time pad ciphers have been used by nations for critical diplomatic and military communication, but the problems of secure key distribution have made them impractical for most applications. 
%%% DETALHAR SEGREDO PERFEITO


\end{document}