\documentclass{article}
\usepackage[utf8]{inputenc}
\usepackage[brazilian]{babel}
\usepackage{tgbonum}
\usepackage{tikz, amsmath}
\usetikzlibrary{arrows, arrows.meta, snakes}


\title{Protocolo Diffie–Hellman para Troca de Chaves}
\author{Jader Martins\\
        Igor Figueira}
\date{\today}

\begin{document}

\maketitle

\begin{abstract}
    O esquema de troca de chaves proposto por Whitfield Diffie e
    Martin Hellman em 1976 é protocolo para realizar
    o intercambio de chaves criptograficas por meio de um canal público
    de forma segura\cite{Diffie}. Este é um dos primeiros protocolos de
    chave publica, foi originalmente conceitualizado por Ralph Merkle
    e sua descrição formal apresentada posteriormente pelos autores que
    nomeam o protocolo \cite{Merkle}.
\end{abstract}

\section{Introdução}%
\label{sec:introducao}
A comunicação secreta é o estabelecimento de um meio de comunicação entre pares
tal que uma terceira parte, também chamado adversário, tenha acesso mínimo ao
conteúdo que esta sendo intercambiado nesse meio.

Tal atividade e seu aprimoramento datam da grécia
antiga\cite{katz2014introduction}, em grande parte, esses segredos eram
estabelecidos pela substituição trivial de letras, cifração, da mensagem em que
apriori eram estabelecidas as regras de substituição e como seria recuperado a
mensagem original. Esses esquemas posteriormente apresentaram diversas
limitações, pois conhecendo o esquema, se tornava trivial a decifração da
mensagem.

Com o surgimento de máquinas automáticas e subsequentemente computadores, a
comunicação segura precisou ser reinventada\cite{singh1999code}, tais mecanismo
permitiam facilmente quebrar esquemas antigos porém propiciaram novas formas de
se codificar mensagens secretas\cite{Diffie}.

%%% FALAR DE SEGREDO PERFEITO E LIMITACOES E DEPOIS INTRODUZIR CHAVES SIMETRICAS

Inicialmente a comunicação segura entre pares requeria que de antemão fossem
trocadas chaves por um meio fisico seguro, usualmente, cartas com a chave
anotadas eram transportadas por entregadores de confiança, esses usavam de
protocolos caros e complexos\cite{matsumoto1987key,Merkle}.

Tal ritual seria impraticável para grupos que detinham pouco poder economico,
surgindo então a busca por formas praticáveis de se distribuir chaves para o
estabelecimento de comunicação segura, dessa busca surgiram os trabalhos de
Ralph Merkle, Whitfield Diffie e Martin Hellman.

O protocolo Diffie-Hellman forneceu a primeira solução viável ao problema
de distribuição de chaves, permitindo duas partes, que inicialmente não se
conhecem ou trocaram qualquer material, estabelecer um segredo compartilhado
(uma chave de encriptação) em um canal aberto, essa chave pode então
subsequentemente ser usada para criptografia simetrica. A segurança se
estabelece pela intratabilidade do problema de Diffie-Hellman e computação de
logarítmos discretos\cite{katz1996handbook}.

\section{O Protocolo}%
\label{sec:o_protocolo}

A versão básica do protocolo, provê proteção na forma de gerar uma
chave em comum entre os pares, sendo que um adversário passivo não a obtenha,
porém o protocolo não garante proteção contra um adversário ativo, esse capaz
de interceptar, modificar ou injetar mensagens. Nenhuma das partes tem
garantias da identidade da fonte da mensagem recebida ou da parte que saberá
a chave resultante\cite{katz1996handbook}.

Posteriormente o protocolo foi aprimorado para resistir a adversários ativos.
%%FALAR DE FORMA BREVE COMO RESISTE A ADVERSARIOS ATIVOS

\subsection{Exemplo Ilustrativo}%
\label{sub:exemplo}
Para o exemplo ilustrativo, supomos que dois entes, Alice e Bob, querem pintar
seus quartos de uma mesma cor sem que Carol, que busca impedir essa amizade,
possa descobrir tal cor. Analogamente, as cores são chaves, e a separação das
cores, tem alto custo, o que em condições ideáis inviabiliza a Carol de obter
as cores secretas e a cor compartilhada.

O processo é iniciado com Alice e Bob, primeiramente eles definem
uma cor em comum arbitrária, no caso da figura \ref{fig:diagram}, rosa, essa
a qual não é preciso ser mantida em segredo de Carol. Em seguida cada um deles
escolhe uma cor secreta para si, no exemplo, Alice escolhe amarelo e Bob
escolhe azul, então essas cores são combinadas as cores que ambos compartilham,
rosa. Para Alice é gerada a cor laranja e para Bob é gerada a cor roxa.

Aqui está a parte essencial do processo, essa terceira cor é trocada entre
Alice e Bob em um meio público, por exemplo, no parquinho onde Carol também
brinca. Mesmo que Carol saiba a cor inicial, rosa, e a cor que foi trocada no
parquinho, laranja e roxo, Carol não conseguirá obter as cores secretas de
Alice e Bob. Alice obtendo a cor de Bob e Bob de Alice, esses a adicionam sua
cor secreta inicial, gerando assim a cor marrom identica para ambos.

No cenário ideal, apenas Bob e Alice terão a cor marrom em comum, podendo assim
usa-las exclusivamente, sem que Carol a descubra. Voltando a analogia inicial,
a cor rosa seria uma chave compartilhada públicamente entre Alice e Bob, as
cores secretas seriam as chaves exclusivas de Alice e Bob, o valor resultante
da operação de ambos poderia ser compartilhado em um canal qualquer que a
construção para a chave final ainda dependeria das chaves exclusivas. A cor
marrom, por fim, seria a chave para encriptar as mensagens trocadas que apenas
Alice e Bob teriam, dado que seria computacionalmente inviável descobrir as
chaves secretas.


\begin{figure}[htpb]
    \centering
    \begin{tikzpicture}[node distance=0.7cm]

\tikzstyle{square}=[draw,minimum size=20pt]
\tikzstyle{wave}=[draw,-latex',snake=snake,segment amplitude=.4mm,segment length=3mm,line after snake=1.5mm]
\tikzstyle{symbol}=[node distance=0.7cm]
\tikzstyle{arrow}=[arrows={{Latex[scale=0.5]}-}, thick]

\node [square, fill=red!25, label=above:Alice]   (b1) at (-2.5,0) {};
\node [square, fill=red!25, label=above:Bob]  (b2) at (2.5,0) {};

\path (b1) --  node [midway]{Cor em comum} (b2);

\node [symbol]  (p1) [below of=b1] {+};
\node [symbol]  (p2) [below of=b2] {+};

\node [square, fill=yellow!25](b3) [below of=p1] {};
\node [square, fill=blue!25](b4) [below of=p2] {};

\path (b3) --  node [midway]{Cores secretas} (b4);

\node [symbol]  (p3) [below of=b3] {=};
\node [symbol]  (p4) [below of=b4] {=};

\node [square, fill=orange!50](b5) [below of=p3] {};
\node [square, fill=purple!50](b6) [below of=p4] {};

\node [symbol]  (p5) [below of=b5] {};
\node [symbol]  (p6) [below of=b6] {};

\node [square, fill=purple!50](b7) [below of=p5] {};
\node [square, fill=orange!50](b8) [below of=p6] {};

\path [wave](b5) --  node [above=1.5em] {Transporte público} (b8);
\path [wave](b6) --  node [text width=3cm,below=1em] {(separação das cores é cara)} (b7);

\node [symbol]  (p7) [below of=b7] {+};
\node [symbol]  (p8) [below of=b8] {+};

\node [square, fill=yellow!25](b9) [below of=p7] {};
\node [square, fill=blue!25](b10) [below of=p8] {};

\path (b9) --  node [midway]{Cores secretas} (b10);

\node [symbol]  (p9) [below of=b9] {=};
\node [symbol]  (p10) [below of=b10] {=};

\node [square, fill=brown!75](b9) [below of=p9] {};
\node [square, fill=brown!75](b10) [below of=p10] {};

\path (b9) --  node [midway]{Segredo compartilhado} (b10);
\end{tikzpicture}

    \caption{Exemplo ilustrativo do protocolo.}%
    \label{fig:diagram}
\end{figure}

\subsection{Descrição formal}%
\label{sub:descricao_formal}



\bibliographystyle{unsrt}
\bibliography{sample}
\end{document}
